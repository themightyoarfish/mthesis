\chapter*{Zusammenfassung}

\begin{otherlanguage}{german}
     Künstliche neuronale Netze haben sich zum populärsten Modelltyp für
     Aufgaben des maschinellen Lernens entwickelt, beispielsweise
     Bildklassifikation, Segmentierung, Video- und Audioverständnis oder
     Zeitreihenvorhersage. Mit immer weiter wachsenden Rechenressourcen sowie
     Fortschritten in der verfügbaren Softwareinfrastruktur vergrößern sich die
     trainierbaren Modelle immer weiter. Nichtsdestoweniger ist es nicht
     unüblich, dass das Training eines Modells Tage oder Wochen in Anspruch
     nimmt, selbst auf hochleistungsfähiger Hardware. Es gibt offensichtliche
     Gründe---zum Beispiel die Schwierigkeit, das Training mit den
     erfolgreichsten Algorithmen zu parallelisieren---gibt es möglicherweise
     weitere Effizienzverluste durch fehlendes Verständins der
     Trainingsdynamiken, was durch teure Parametersuche umgangen wird.

     Der pragmatische Ansatz, die Black Box des Deep Learning zu öffnen, ist das
     Experimentieren und Beobachten. Forscher führen Experimente durch, um
     Einsichten in den Trainingsprozess zu erhalten, welche dann verwendet
     werden können, um das Training zu leiten. Die Geschwindigkeit, mit der
     solche Experimente durchgeführt werden können, hängt auch von den verfügbaren
     Werkzeugen ab.

     Diese Arbeit stellt eine Softwarebibliothek vor, die das Experimentieren
     mit Online-Metriken vereinfacht, welche letztendlich das Training
     beschleunigen könnten. Jene Bibliothek kann dann auch verwendet werden,
     gefundene Metriken live für beliebige Modelle zu überwachen. Die Software
     wird weiterhin verwendet, um solcherlei bisher unerforschte Metriken für
     eine Auswahl an Problemen zu erarbeiten. Zudem werden unter Zuhilfenahme
     der Software bekannte Resultate und Behauptungen validiert, um die
     Nützlichkeit für Unterschungen im Deep Learning zu demonstrieren.

\end{otherlanguage}
